\documentclass[10pt]{article}

\usepackage {
  amsmath,
  amssymb,
  amsthm,
  array,
  graphicx,
  multicol
}

\usepackage[colorlinks=true]{hyperref}
\usepackage[dvipsnames]{xcolor}
\usepackage[english]{babel}
\usepackage[margin=1in]{geometry}
\usepackage[utf8]{inputenc}
\usepackage{color}
\usepackage{circuitikz}

\hypersetup {
  citecolor = ForestGreen,
  filecolor = Plum,
  linkcolor = NavyBlue,
  urlcolor = RubineRed
}

\newcommand{\C}{\mathbb{C}}
\newcommand{\F}{\mathbb{F}}
\newcommand{\Q}{\mathbb{Q}}
\newcommand{\Z}{\mathbb{Z}}
\newcommand{\R}{\mathbb{R}}
\newcommand{\N}{\mathbb{N}}
\newcommand{\CO}{\mathcal{O}}
\newcommand{\CC}{\mathcal{C}}
\newcommand{\CU}{\mathcal{U}}
\newcommand{\spacing}{\vspace*{1\baselineskip}}

\title{MATH 240: Discrete Structures - Assignment 4}
\author{Liam Scalzulli\\
\href{mailto:liam.scalzulli@mail.mcgill.ca}{liam.scalzulli@mail.mcgill.ca}}
\date{\today}

\begin{document}
\maketitle

\subsection*{Problem 1}

\subsubsection*{(a)}

\begin{proof}
  (By Induction)
  \spacing

  \noindent
  \underline{Base case:} k = 0
  \spacing

  \noindent
  $${n \choose n} = {n + 1 \choose n + 1} = 1$$
  \spacing

  \noindent
  \underline{Inductive step:}
  \spacing

  \noindent
  Assume $\sum_{i = 0}^{k}{n + i \choose n} = {n + k + 1 \choose n + 1}$ for some $k$, then:

  \begin{align*}
   \sum_{i = 0}^{k + 1}{n + i \choose n} &= \sum_{i = 0}^{k}{n + i \choose n} + {n + k + 1 \choose n} \\
     &= {n + k + 1 \choose n + 1} + {n + k + 1 \choose n} \\
     &= {n + k + 2 \choose n + 1} \\
     &= {n + (k + 1) + 1 \choose n + 1}
  \end{align*}

  \noindent
  which is what we wanted to show.
\end{proof}

\subsubsection*{(b)}

\begin{proof}
  (By Combinatorial Proof)
  \spacing

  \noindent
  Choose any integers $n$, $k$ and let $S$ be a set such that $|S| = n + k + 1$.
  \spacing

  \noindent
  The RHS counts the number of $n + 1$ element subsets of S.
  \spacing

  \noindent
  The LHS counts the same thing in a different way. Remove $k + 1$ elements
  from $S$, then count the number of $n$ element subsets of $S$, repeat this
  until we've added each removed element back into $S$. We arrive at the same conclusion.
\end{proof}

\newpage
\subsection*{Problem 2}

The bulk of the work for this problem lay in figuring out
what our 'pigeonholes' will be. 
\spacing

\noindent
Since the we have points of the form $(x, y, z)$, the mid-point, or 'average', of any two 
points will be of the form $(\frac{x_{1} + x_{2}}{2}, \frac{y_{1} + y_{2}}{2}, \frac{z_{1} + z_{2}}{2})$.
We want the mid-point to be integral, that is, $\frac{x_{1} + x_{2}}{2} \in \Z$, 
$\frac{y_{1} + y_{2}}{2} \in \Z$ and $\frac{z_{1} + z_{2}}{2} \in \Z$, so we notice that the numerator
values must be of the same parity.
\spacing

\noindent
Our pigeonholes can thus be the sets of points $(x, y, z) \in \Z \times \Z \times \Z$ according to their parity:
\vspace*{-15pt}

\begin{align*}
  H_{EOO} &= \{\text{($x$, $y$, $z$)} \in \Z \times \Z \times \Z : (x \text{ even, } y \text{ odd,  } z \text{ odd})\}  \\
  H_{EEO} &= \{\text{($x$, $y$, $z$)} \in \Z \times \Z \times \Z : (x \text{ even, } y \text{ even, } z \text{ odd})\}  \\
  H_{EEE} &= \{\text{($x$, $y$, $z$)} \in \Z \times \Z \times \Z : (x \text{ even, } y \text{ even, } z \text{ even})\} \\
  H_{OEE} &= \{\text{($x$, $y$, $z$)} \in \Z \times \Z \times \Z : (x \text{ odd,  } y \text{ even, } z \text{ even})\} \\
  H_{EOE} &= \{\text{($x$, $y$, $z$)} \in \Z \times \Z \times \Z : (x \text{ even, } y \text{ odd,  } z \text{ even})\} \\
  H_{OOE} &= \{\text{($x$, $y$, $z$)} \in \Z \times \Z \times \Z : (x \text{ odd,  } y \text{ odd,  } z \text{ even})\} \\
  H_{OOO} &= \{\text{($x$, $y$, $z$)} \in \Z \times \Z \times \Z : (x \text{ odd,  } y \text{ odd,  } z \text{ odd})\}  \\
  H_{OEO} &= \{\text{($x$, $y$, $z$)} \in \Z \times \Z \times \Z : (x \text{ odd,  } y \text{ even, } z \text{ odd})\}
\end{align*}

\noindent
Since we take 9 points of the form $(x, y, z) \in \Z \times \Z \times \Z$, note that at least $2$ of them will land
in the same 'pigeonhole' by the Pigeonhole Principle, and thus whose mid-point will be integral, giving us our answer.

\newpage
\subsection*{Problem 3}

\subsubsection*{(a)}

We have three base cases:

\begin{align*}
  B_{1} &= 1 \\
  B_{2} &= 1 \\
  B_{3} &= 2
\end{align*}

\noindent
For which we can deduce the recurrence as being:

$$B_{n} = B_{n - 1} + B_{n - 2}, \text{ for }n \ge 3$$

\subsubsection*{(b)}

\begin{proof}
  (By Induction)
  \spacing

  \noindent
  \underline{Base case:} $n = 4$
  \spacing

  \noindent
  $$B_{4} = B_{3} + B_{2} = 2 + 1 = 3$$
  which is divisible by 3.
  \spacing

  \noindent
  \underline{Inductive step:}
  \spacing

  \noindent
  Assume $B_{4k}$ is divisible by 3 for some $k > n \in \N$.
  \spacing

  \noindent
  We have $B_{4k} = 3l \text{ for some } l \in Z$.

  \begin{align*}
    B_{4k} = B_{4(k + 1)} &= B_{4k + 4} \\
      &= B_{4k + 3} + B_{4k + 4} && \text{[By RR]} \\
      &= 2B_{4k + 2} + B_{4k + 1} \\
      &= 2(B_{4k + 1} + B_{4k}) + B_{4k + 1} \\
      &= 3B_{4k + 1} + 2B_{4k} \\
      &= 3B_{4k + 1} + 2(3l) && \text{[By IH]} \\
      &= 3(B_{4k + 1} + 2l)
  \end{align*}
  
  \noindent
  Which is divisible by 3, therefore proving that if $n$ is divisible by $4$, $B_{n}$ is divisible by 3,
  by mathematical induction.
\end{proof}

\newpage
\subsection*{Problem 4}

\subsubsection*{(a)}

I approach this by setting up the homegeneous recurrence relation, solving for the characteristic
polynomial, solving for roots and then solving for the remaining constants $a_{1}$ and $a_{2}$.

\begin{align*}
  g_{n} - 4g_{n - 1} + 3g_{n - 2} = 0 \\
  = r^{n} - 4r^{n - 1} + 3r^{n - 2} = 0 \\
  = r^2 - 4r + 3 = 0 \\
  = (r - 1)(r - 3) = 0
\end{align*}

\noindent
So after this we have roots of the characteristic polynomial = $\{1, 3\}$, therefore we need to solve
for constants $a_{1}$ and $a_{2}$ such that $g_{n} = a_{1}(1)^n + a_{2}(3)^n$.
\spacing

\noindent
We have base cases $g_{0} = 1$ and $g_{1} = 7$, so:

\begin{align*}
  1 = a_{1}(1)^0 + a_{2}(3)^0 \\
  7 = a_{1}(1)^1 + a_{2}(3)^1 \\
\end{align*}

\noindent
Solving for $a_{1}$ and $a_{2}$:

\begin{align*}
  1 - a_{2} = a_{1} \\
  7 = 1 - a_{2} + 3a_{2} \\
  2a_{2} = 6 \\
  \underline{a_{2} = 3} \\
  1 - 3 = a_{1} \\
  \underline{a_{1} = -2}
\end{align*}

\noindent
So we have $g_{n} = -2(1)^n + 3(3)^n$.

\subsubsection*{(b)}

In order to figure out what the recurrence relation might be, its useful to generate
some base cases.
\spacing

\noindent
Let $f_{n}$ be the function that computes the number of sequences of length $n$ over the alphabet
$\{0, 1, 2\}$ that do not contain consecutive even numbers.
\spacing

\noindent
For $f_{0}$, we know this to be $1$, as the empty string is a valid string given the criteria. We also have
$f_{1} = 3$, as we can clearly see the only sequences of length $1$ are $\{0, 1, 2\}$, which are valid sequences.
\spacing

\noindent
Moreover, we can see that $f_{2} = 5$ since we remove the sequences $\{02, 22, 00, 20\}$ 
from the set $\{10, 11, 01, 02, 22, 00, 20, 12, 21\}$, giving us enough information to deduce
a recurrence relation:

\begin{align*}
  f_{0} &= 1 \\
  f_{1} &= 3 \\
  f_{2} &= 5 \\
  f_{n} &= f_{n - 1} + 2f_{n - 2}
\end{align*}

\noindent
Solving for the characteristic polynomial:

\begin{align*}
  f_{n} - f_{n - 1} - 2f_{n - 2} = 0 \\
  = r^{n} - r^{n - 1} - 2r^{n - 2} = 0 \\
  = r^2 - r - 2 = 0 \\
  = (r + 1)(r - 2) = 0
\end{align*}

\noindent
Thus, we have roots of the characteristic polynomial = $\{-1, 2\}$, therefore we need to solve
for constants $a_{1}$ and $a_{2}$ such that $f_{n} = a_{1}(2)^n + a_{2}(-1)^n$.
\spacing

\noindent
We use the base cases $f_{0} = 1$ and $f_{0} = 3$, so:

\begin{align*}
  1 = a_{1}(2)^0 + a_{2}(-1)^0 \\
  3 = a_{1}(2)^1 + a_{2}(-1)^1 \\
\end{align*}

\noindent
Solving for $a_{1}$ and $a_{2}$:

\begin{align*}
  1 - a_{2} = a_{1} \\
  3 = 2(1 - a_{2}) - a_{2} \\
  3 = 2 - 2a_{2} - a_{2} \\
  1 = -3a_{2} \\
  \underline{a_{2} = \frac{1}{3}} \\
  a_{1} + 1 + \frac{1}{3} \\
  \underline{a_{1} = \frac{4}{3}}
\end{align*}

\noindent
So we have $f_{n} = \frac{4}{3}(2)^n - \frac{1}{3}(-1)^n$.

\end{document}