\documentclass[10pt]{article}

\usepackage {
  amsmath,
  amssymb,
  amsthm,
  array,
  graphicx,
  multicol
}

\usepackage[colorlinks=true]{hyperref}
\usepackage[dvipsnames]{xcolor}
\usepackage[english]{babel}
\usepackage[margin=1in]{geometry}
\usepackage[utf8]{inputenc}
\usepackage{color}
\usepackage{circuitikz}

\hypersetup {
  citecolor = ForestGreen,
  filecolor = Plum,
  linkcolor = NavyBlue,
  urlcolor = RubineRed
}

\newcommand{\C}{\mathbb{C}}
\newcommand{\F}{\mathbb{F}}
\newcommand{\Q}{\mathbb{Q}}
\newcommand{\Z}{\mathbb{Z}}
\newcommand{\R}{\mathbb{R}}
\newcommand{\N}{\mathbb{N}}
\newcommand{\CO}{\mathcal{O}}
\newcommand{\CC}{\mathcal{C}}
\newcommand{\CU}{\mathcal{U}}
\newcommand{\spacing}{\vspace*{1\baselineskip}}

\title{MATH 240: Discrete Structures - Assignment 4}
\author{Liam Scalzulli\\
\href{mailto:liam.scalzulli@mail.mcgill.ca}{liam.scalzulli@mail.mcgill.ca}}
\date{\today}

\begin{document}
\maketitle

\subsection*{Problem 1}

\subsubsection*{(a)}

\subsubsection*{(b)}

\newpage
\subsection*{Problem 2}

The bulk of the work for this problem lay in figuring out
what our 'pigeonholes' will be. 
\spacing

\noindent
Since the we have points of the form $(x, y, z)$, the mid-point, or 'average', of any two 
points will be of the form $(\frac{x_{1} + x_{2}}{2}, \frac{y_{1} + y_{2}}{2}, \frac{z_{1} + z_{2}}{2})$.
We want the mid-point to be integral, that is, $\frac{x_{1} + x_{2}}{2} \in \Z$, 
$\frac{y_{1} + y_{2}}{2} \in \Z$ and $\frac{z_{1} + z_{2}}{2} \in \Z$, so we notice that the numerator
values must be of the same parity.
\spacing

\noindent
Our pigeonholes can thus be the sets of points $(x, y, z) \in \Z \times \Z \times \Z$ according to their parity:
\vspace*{-15pt}

\begin{align*}
  H_{EOO} &= \{\text{($x$, $y$, $z$)} \in \Z \times \Z \times \Z : (x \text{ even, } y \text{ odd,  } z \text{ odd})\}  \\
  H_{EEO} &= \{\text{($x$, $y$, $z$)} \in \Z \times \Z \times \Z : (x \text{ even, } y \text{ even, } z \text{ odd})\}  \\
  H_{EEE} &= \{\text{($x$, $y$, $z$)} \in \Z \times \Z \times \Z : (x \text{ even, } y \text{ even, } z \text{ even})\} \\
  H_{OEE} &= \{\text{($x$, $y$, $z$)} \in \Z \times \Z \times \Z : (x \text{ odd,  } y \text{ even, } z \text{ even})\} \\
  H_{EOE} &= \{\text{($x$, $y$, $z$)} \in \Z \times \Z \times \Z : (x \text{ even, } y \text{ odd,  } z \text{ even})\} \\
  H_{OOE} &= \{\text{($x$, $y$, $z$)} \in \Z \times \Z \times \Z : (x \text{ odd,  } y \text{ odd,  } z \text{ even})\} \\
  H_{OOO} &= \{\text{($x$, $y$, $z$)} \in \Z \times \Z \times \Z : (x \text{ odd,  } y \text{ odd,  } z \text{ odd})\}  \\
  H_{OEO} &= \{\text{($x$, $y$, $z$)} \in \Z \times \Z \times \Z : (x \text{ odd,  } y \text{ even, } z \text{ odd})\}
\end{align*}

\noindent
Since we take 9 points of the form $(x, y, z) \in \Z \times \Z \times \Z$, note that at least $2$ of them will land
in the same 'pigeonhole' by the Pigeonhole Principle, and thus whose mid-point will be integral, giving us our answer.

\newpage
\subsection*{Problem 3}

\subsubsection*{(a)}

We have two base cases:

$$B_{1} = B_{2} = 1$$

\noindent
For which we can deduce the recurrence as being:

$$B_{n} = B_{n - 1} + B_{n - 2}, \text{ for }n \ge 3$$

\subsubsection*{(b)}

\begin{proof}
  (By Induction)
  \spacing

  \noindent
  \underline{Base case:} $n = 4$
  \spacing

  \noindent
  $$B_{4} = B_{3} + B_{2} = 2 + 1 = 3$$
  which is divisible by 3.
  \spacing

  \noindent
  \underline{Inductive step:}
  \spacing

  \noindent
  Assume $B_{4k}$ is divisible by 3 for some $k > n \in \N$.
  \spacing

  \noindent
  We have $B_{4k} = 3l \text{ for some } l \in Z$.

  \begin{align*}
    B_{4k} = B_{4(k + 1)} &= B_{4k + 4} \\
      &= B_{4k + 3} + B_{4k + 4} && \text{[By RR]} \\
      &= 2B_{4k + 2} + B_{4k + 1} \\
      &= 2(B_{4k + 1} + B_{4k}) + B_{4k + 1} \\
      &= 3B_{4k + 1} + 2B_{4k} \\
      &= 3B_{4k + 1} + 2(3l) && \text{[By IH]} \\
      &= 3(B_{4k + 1} + 2l)
  \end{align*}
  
  \noindent
  Which is divisible by 3, therefore proving that if $n$ is divisible by $4$, $B_{n}$ is divisible by 3,
  by mathematical induction.
\end{proof}

\newpage
\subsection*{Problem 4}

\subsubsection*{(a)}

\subsubsection*{(b)}

\end{document}