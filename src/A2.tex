\documentclass[10pt]{article}

\usepackage {
  amsmath,
  amssymb,
  amsthm,
  array,
  graphicx,
  multicol,
  longdivision
}

\usepackage[colorlinks=true]{hyperref}
\usepackage[dvipsnames]{xcolor}
\usepackage[english]{babel}
\usepackage[margin=1in]{geometry}
\usepackage[utf8]{inputenc}
\usepackage{color}
\usepackage{circuitikz}

\hypersetup {
  citecolor = ForestGreen,
  filecolor = Plum,
  linkcolor = NavyBlue,
  urlcolor = RubineRed
}

\newcommand{\C}{\mathbb{C}}
\newcommand{\F}{\mathbb{F}}
\newcommand{\Q}{\mathbb{Q}}
\newcommand{\Z}{\mathbb{Z}}
\newcommand{\R}{\mathbb{R}}
\newcommand{\N}{\mathbb{N}}
\newcommand{\CO}{\mathcal{O}}
\newcommand{\CC}{\mathcal{C}}
\newcommand{\CU}{\mathcal{U}}
\newcommand{\spacing}{\vspace*{1\baselineskip}}

\title{MATH 240: Discrete Structures - Assignment 2}
\author{Liam Scalzulli\\
\href{mailto:liam.scalzulli@mail.mcgill.ca}{liam.scalzulli@mail.mcgill.ca}}
\date{\today}

\begin{document}
\maketitle

\subsection*{Problem 1}

\subsubsection*{(a)}

\begin{proof}
  Suppose $P(x) = 2x^2 - 4x + 3$ where $x \in \R$.

  \begin{align*}
    2x^2 - 4x + 3 &= 2(x^2 - 2x + 1) + 1 \\
      &= 2((x - 1)(x - 1)) + 1 \\
      &= 2(x - 1)^2 + 1
  \end{align*}

  \noindent
  Notice that $x^2 \ge 0 \: \forall \: x \in \R$.
  \spacing

  \noindent
  It follows that $2(x - 1)^2 + 1 > 0 \: \forall \: x \in \R$.
  \spacing

  \noindent
  So $P(x) > 0 \: \forall \: x \in \R$.
\end{proof}

\subsubsection*{(b)}

\begin{proof}
  Suppose $x \in \Z$ is odd.

  \begin{align*}
    x &= 2k + 1, \: k \in Z \\
      &= 1 \times (2k + 1) \\
      &= (k + 1 + k)(k + 1 - k)
  \end{align*}

  \noindent
  Let $a = k + 1$, $b = k$. \spacing

  \noindent
  Then $2k + 1 = (a + b)(a - b) = a^2 - b^2.$
\end{proof}

\newpage
\subsection*{Problem 2}

\subsubsection*{(a)}

\begin{proof}
  Suppose $x \in \Z$ is even.
  \spacing

  \noindent
  Then we can write $x = 2k$, $k \in \Z$.

  \begin{align*}
    x^3 - 2x + 3 &= (2k)^3 - 2(2k) + 3 \\
      &= 8k^3 - 4k + 3 \\
      &= 8k^3 - 4k + 2 + 1 \\
      &= 2(4k^3 - 2k + 1) + 1
  \end{align*}

  \noindent
  Let $l = 4k^3 - 2k + 1$, $l \in \Z$.
  \spacing

  \noindent
  Then $x^3 - 2x + 3 = 2l + 1$, and is odd by definition.
\end{proof}

\subsubsection*{(b)}

\begin{proof}
  Suppose $log_{2}5$ is rational.
  \spacing

  \noindent
  Then we can write $log_{2}5 = \frac{m}{n}$, where $m, \: n \in \Z$.

  \begin{align*}
    5 &= 2^{(\frac{m}{n})} \\
    5^n &= 2^m
  \end{align*}

  \noindent
  Since $5$ is odd and all $5^n, \: n \in \N$ are odd by induction and $2$ is even 
  and all $2^m, \: m \in \N$ are even by induction, an odd number can never equal an even number, 
  we've reached a contradiction, hence $log_{2}5$ is irrational.
\end{proof}

\newpage
\subsection*{Problem 3}

\subsubsection*{(a)}

\begin{proof}
  Let $P(n) = \frac{1}{1 \cdot 3} + \frac{1}{3 \cdot 5} + \frac{1}{5 \cdot 7} +
  ... + \frac{1}{(2n - 1) \cdot (2n + 1)} = \frac{n}{2n + 1}$.
  \spacing

  \noindent
  \underline{Base case}: $n = 1$
  $$ \frac{1}{(1 \cdot 3)} = \frac{1}{3} $$
  $$ \frac{n}{2n + 1} = \frac{1}{2(1) + 1} = \frac{1}{3} $$
  \spacing

  \noindent
  \underline{Induction step}:
  Assume $P(n)$ holds for some $n \le k$.

  $$ \text{Then } \frac{1}{1 \cdot 3} + \frac{1}{3 \cdot 5} +
  \frac{1}{5 \cdot 7} + ... + \frac{1}{(2k - 1) \cdot (2k + 1)} = \frac{k}{2k + 1} $$

  \begin{align*}
    \frac{k}{2k + 1} + \frac{1}{(2(k + 1) - 1)(2(k + 1) + 1)} &=
    \frac{k(2k + 3) + 1}{(2k + 1)(2k + 3)} && \text{[Common Denominator]} \\
      &= \frac{2k^2 + 3 + 1}{(2k + 1)(2k + 3)} && \text{[Expand $k(2k + 3) + 1$]} \\
      &= \frac{(2k + 1)(k + 1)}{(2k + 1)(2k + 3)} && \text{[Factor $2k^2 + 3 + 1$]} \\
      &= \frac{k + 1}{2k + 3}
  \end{align*}

  \noindent
  Notice that $\frac{k + 1}{2k + 3}$ is just $P(k + 1)$, $\therefore$ we've shown that $k \implies k + 1$
  by mathematical induction.
\end{proof}

\subsubsection*{(b)}

\begin{proof}
  Let $P(n) = \overline{A_{1} \cup A_{2} \cup ... \cup A_{n}} = \overline{A_{1}}
  \cap \overline{A_{2}} \cap ... \cap \overline{A_{n}}$.
  \spacing

  \noindent
  \underline{Base case}: $n = 2$
  $$\overline{A_{1} \cup A_{2}} = \overline{A_{1}} \cap \overline{A_{2}}$$
  Which is true by DeMorgan's Law.
  \spacing

  \noindent
  \underline{Induction step}:
  Assume $P(n)$ holds for some $n \le k$.

  $$\text{Then } \overline{A_{1} \cup A_{2} \cup ... \cup A_{k}} =
  \overline{A_{1}} \cap \overline{A_{2}} \cap ... \cap \overline{A_{k}}$$

  \begin{align*}
  \overline{(A_{1} \cup A_{2} \cup ... \cup A_{k}) \cup A_{k + 1}} &=
  \overline{(A_{1} \cup A_{2} \cup ... \cup A_{k})} \cap \overline{A_{k+1}} && \text{[De Morgan]}\\
    &= (\overline{A_{1}} \cap \overline{A_{2}} \cap ... \cap \overline{A_{k}})
      \cap \overline{A_{k + 1}} && \text{[Induction Hypothesis]} \\
    &= \overline{A_{1}} \cap \overline{A_{2}} \cap ...
      \cap \overline{A_{k}} \cap \overline{A_{k + 1}} && \text{[Associativity]}
  \end{align*}

  \noindent
  Therefore we've shown that we can reach $k + 1$ from $k$ by mathematical induction.
\end{proof}

\newpage
\subsection*{Problem 4}

\begin{proof}
  Let $a$, $b$, $c$ $\in \N$.
  \spacing

  \noindent
  Suppose $a \mid (b + c)$ and $gcd(b, c) = 1$.
  \spacing

  \noindent
  We can then write $ma = b + c$, $m \in \N$, and by Bézout's Identity we have $1 = bu + cv$
  where $b, c \in \N$.
  \spacing

  \noindent
  Then:
  $$b = ma - c$$
  $$c = ma - b$$
  \spacing

  \noindent
  \underline{Case 1}: Showing $gcd(a, b) = 1$
  \begin{align*}
    1 &= bu + cv \\
    1 &= bu + (ma - b)v \\
    1 &= mav - bv + bu \\
    1 &= a(mv) + b(u - v)
  \end{align*}
  Since $mv$, $u - v \in \N$, by Bézout's Identity $gcd(a, b) = 1$.
  \spacing

  \noindent
  \underline{Case 2}: Showing $gcd(a, c) = 1$
  \begin{align*}
    1 &= bu + cv \\
    1 &= (ma - c)u + cv \\
    1 &= uma - cu + cv \\
    1 &= a(um) + c(v - u)
  \end{align*}
  Since $um$, $v - u \in \N$, by Bézout's Identity $gcd(a, c) = 1$.
\end{proof}

\end{document}
