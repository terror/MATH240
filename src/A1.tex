\documentclass[10pt]{article}

\usepackage {
  amsmath,
  amssymb,
  amsthm,
  array,
  graphicx,
  multicol,
  longdivision
}

\usepackage[colorlinks=true]{hyperref}
\usepackage[dvipsnames]{xcolor}
\usepackage[english]{babel}
\usepackage[margin=1in]{geometry}
\usepackage[utf8]{inputenc}
\usepackage{color}
\usepackage{circuitikz}

\hypersetup {
  citecolor = ForestGreen,
  filecolor = Plum,
  linkcolor = NavyBlue,
  urlcolor = RubineRed
}

\newcommand{\C}{\mathbb{C}}
\newcommand{\F}{\mathbb{F}}
\newcommand{\Q}{\mathbb{Q}}
\newcommand{\Z}{\mathbb{Z}}
\newcommand{\R}{\mathbb{R}}
\newcommand{\N}{\mathbb{N}}
\newcommand{\CO}{\mathcal{O}}
\newcommand{\CC}{\mathcal{C}}
\newcommand{\CU}{\mathcal{U}}
\newcommand{\spacing}{\vspace*{1\baselineskip}}

\title{MATH 240: Discrete Structures - Assignment 1}
\author{Liam Scalzulli\\
\href{mailto:liam.scalzulli@mail.mcgill.ca}{liam.scalzulli@mail.mcgill.ca}}
\date{\today}

\begin{document}
\maketitle

\subsection*{Problem 1}

To do this we can simplify the right-hand side using set identities,
ultimately showing that it is equivalent to the left-hand side.

\begin{align*}
  C \setminus B &= (C \setminus B) \setminus (B \setminus A) \\
    &= (C \cap \overline{B}) \cap \overline{(B \cap \overline{A})}
      && \text{[Definition } A \setminus B = A \cap \overline{B}] \\
    &= (C \cap \overline{B}) \cap (\overline{B} \cup A) && \text{[De Morgan's
      Law: } \overline{A \cap B} = \overline{A} \cup \overline{B}] \\
    &= C \cap \overline{B} \cap (\overline{B} \cup A) && \text{[Associative Law: }
      (A \cap B) \cap C = A \cap (B \cap C)]\\
    &= C \cap \overline{B} && \text{[Absorption Law: } A \cap (A \cup B) = A] \\
    &= C \setminus B
\end{align*}

\newpage
\subsection*{Problem 2}

Recall: A tautology is a statement which contains all true in the last column
of its truth table.

\subsubsection*{(a)}

\begin{displaymath}
  \begin{array}{c|c|c|c|c|c|c}
    p & q & \overline{q} & p \implies q & \overline{p \implies q} & \overline{q}
    \land (\overline{p \implies q}) & (\overline{q} \land (\overline{p \implies
    q})) \iff p \\
    \hline
    T & T & F & T & F & F & F \\
    T & F & T & F & T & T & T \\
    F & T & F & T & F & F & T \\
    F & F & T & T & F & F & T \\
  \end{array}
\end{displaymath}

$$\therefore (\overline{q} \land (\overline{p \implies q})) \iff p \textrm{ is
  \emph{not} a tautology.}$$

\subsubsection*{(b)}

\begin{displaymath}
  \begin{array}{c|c|c|c|c|c|c|c}
    p & q & r & q \implies r & p \implies (q \implies r) & p \implies q & (p
    \implies q) \implies r & (p \implies (q \implies r)) \iff ((p \implies q)
    \implies r) \\
    \hline
    T & T & T & T & T & T & T & T \\
    T & T & F & F & F & T & F & T \\
    T & F & T & T & T & F & T & T \\
    T & F & F & T & T & F & T & T \\
    F & T & T & T & T & T & T & T \\
    F & T & F & F & T & T & F & F \\
    F & F & T & T & T & T & T & T \\
    F & F & F & T & T & T & F & F \\
  \end{array}
\end{displaymath}

$$\therefore (p \implies (q \implies r)) \iff ((p \implies q) \implies r)
\textrm{ is \emph{not} tautology.}$$

\newpage
\subsection*{Problem 3}

To prove logical equivalence we can simplify either the left-hand side or the
right-hand side using the laws of logic, ultimately showing that one side is
equivalent to the other.

\subsubsection*{(a)}

\begin{align*}
  (p \iff q) &\equiv (p \implies q) \land (q \implies p) && \text{[Definition }
  (p \iff q) \equiv (p \implies q) \land (q \implies p)] \\
    &\equiv (\overline{p} \lor q) \land (\overline{q} \lor p) && \text{[Definition }
    (p \implies q) \equiv \overline{p} \lor q]
\end{align*}

\begin{align*}
  \overline{p \oplus q} &\equiv \overline{(p \land \overline{q})
  \lor (\overline{p} \land q}) && \text{[Definition } p \oplus q
  \equiv (p \land \overline{q}) \lor (\overline{p} \land q)] \\
    &\equiv \overline{(p \land \overline{q})} \land \overline{(\overline{p}
      \land q)} && \text{[De Morgan's Law: } \overline{p \lor q}
      \equiv \overline{p} \land \overline{q}] \\
    &\equiv (\overline{p} \lor q) \land (q \lor \overline{p})
      && \text{[De Morgan's Law: } \overline{p \land q}
      \equiv \overline{p} \lor \overline{q}] \\
\end{align*}

$$\therefore p \iff q \equiv \overline{p \oplus q}$$

\subsubsection*{(b)}

\begin{align*}
  \overline{p} &\equiv (\overline{p \lor \overline{q}}) \lor (\overline{p}
  \land \overline{q}) \\
    &\equiv (\overline{p} \land q) \lor (\overline{p} \land \overline{q}) &&
      \text{[De Morgan's Law: } \overline{p \lor q} \equiv \overline{p}
      \land \overline{q}] \\
    &\equiv \overline{p} \land (q \lor \overline{q}) && \text{[Distributive Law: }
      p \land (q \lor r) \equiv (p \land q) \lor (p \land r)] \\
    &\equiv \overline{p} \land 1 && \text{[Tautology: } p \lor
      \overline{p} \equiv 1] \\
    &\equiv \overline{p} && \text{[Identity Law: } p \land 1 \equiv p]
\end{align*}

\newpage
\subsection*{Problem 4}

Recall \textbf{NOR}: $(p \downarrow q) \equiv \overline{p} \land \overline{q}
\equiv \overline{(p \lor q)} \text{ *as seen in class}$

\subsubsection*{(a) OR ($p \lor q$)}

\begin{center}
  \begin{circuitikz}
    \draw
    (0,2) node[nor port] (a) {}
    (0,0) node[nor port] (b) {}
    (2,1) node[nor port] (c) {}
    (a.in 1) -- ++(-0.5, 0) node[left] {$p$}
    (a.in 2) -- ++(-0.5, 0) node[left] {$q$}
    (b.in 1) -- ++(-0.5, 0) node[left] {$p$}
    (b.in 2) -- ++(-0.5, 0) node[left] {$q$}
    (a.out) -- (c.in 1) node[above,yshift=7.5mm,xshift=-2.5mm] {$\overline{(p \lor q)}$}
    (b.out) -- (c.in 2) node[below,yshift=-7.5mm,xshift=-2.5mm] {$\overline{(p \lor q)}$}
    [->] (c.out) -- ++(0.5, 0) node[right] {$\overline{\overline{(p \lor q)} \lor
    \overline{(p \lor q)}} \equiv (p \lor q) \land (p \lor q) \equiv p \lor q$};
  \end{circuitikz}
\end{center}

\subsubsection*{(b) AND ($p \land q$)}

\begin{center}
  \begin{circuitikz}
    \draw
    (0,2) node[nor port] (a) {}
    (0,0) node[nor port] (b) {}
    (2,1) node[nor port] (c) {}
    (a.in 1) -- ++(-0.5, 0) node[left] {$p$}
    (a.in 2) -- ++(-0.5, 0) node[left] {$p$}
    (b.in 1) -- ++(-0.5, 0) node[left] {$q$}
    (b.in 2) -- ++(-0.5, 0) node[left] {$q$}
    (a.out) -- (c.in 1) node[above,yshift=7.5mm,xshift=-2.5mm] {$\overline{(p \lor p)}$}
    (b.out) -- (c.in 2) node[below,yshift=-7.5mm,xshift=-2.5mm] {$\overline{(q \lor q)}$}
    [->] (c.out) -- ++(0.5, 0) node[right] {$\overline{\overline{(p \lor p)} \lor
    \overline{(q \lor q)}} \equiv (p \lor p) \land (q \lor q) \equiv p \land q$};
  \end{circuitikz}
\end{center}

\subsubsection*{(c) NOT ($\overline{p}$)}

\begin{center}
  \begin{circuitikz}
    \draw
    (2,1) node[nor port] (a) {}
    (a.in 1) -- ++(-0.5, 0) node[left] {$p$}
    (a.in 2) -- ++(-0.5, 0) node[left] {$p$}
    [->] (c.out) -- ++(0.5, 0) node[right] {$\overline{p \lor p} \equiv
    \overline{p} \land \overline{p} \equiv \overline{p}$};
  \end{circuitikz}
\end{center}

\end{document}
