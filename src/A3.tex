\documentclass[10pt]{article}

\usepackage {
  amsmath,
  amssymb,
  amsthm,
  array,
  graphicx,
  multicol
}

\usepackage[colorlinks=true]{hyperref}
\usepackage[dvipsnames]{xcolor}
\usepackage[english]{babel}
\usepackage[margin=1in]{geometry}
\usepackage[utf8]{inputenc}
\usepackage{color}
\usepackage{circuitikz}

\hypersetup {
  citecolor = ForestGreen,
  filecolor = Plum,
  linkcolor = NavyBlue,
  urlcolor = RubineRed
}

\newcommand{\C}{\mathbb{C}}
\newcommand{\F}{\mathbb{F}}
\newcommand{\Q}{\mathbb{Q}}
\newcommand{\Z}{\mathbb{Z}}
\newcommand{\R}{\mathbb{R}}
\newcommand{\N}{\mathbb{N}}
\newcommand{\CO}{\mathcal{O}}
\newcommand{\CC}{\mathcal{C}}
\newcommand{\CU}{\mathcal{U}}
\newcommand{\spacing}{\vspace*{1\baselineskip}}

\title{MATH 240: Discrete Structures - Assignment 3}
\author{Liam Scalzulli\\
\href{mailto:liam.scalzulli@mail.mcgill.ca}{liam.scalzulli@mail.mcgill.ca}}
\date{\today}

\begin{document}
\maketitle

\subsection*{Problem 1}

\begin{proof}
  Assume $2^n - 1$ is prime. \spacing

  \noindent
  Let $\sigma(n)$ be a function that computes the sum of all divisors of a given
  number $n$.
  \spacing

  \noindent
  Then $\sigma(2^n - 1) = 2^n$ since the only divisors of $2^n - 1$ are $1$ and $2 ^ n - 1$ as a result
  of $2^n - 1$ being prime.
  \spacing

  \noindent
  Moreover, we have $\sigma(2^{n - 1}) = 2^n - 1$ as the divisors of $2^{n - 1}$ are all powers of $2$ up to
  and including $2^{n - 1}$.
  \spacing

  \noindent
  Putting it all together:

  \begin{align*}
    \sigma(2^{n - 1}(2^{n - 1}))) &= \sigma(2^{n - 1})\sigma(2^n - 1) \\
      &= (2^n - 1)(2^n) \\
      &= 2(2^{n - 1})(2^n - 1)
  \end{align*}

  \noindent
  Therefore $2^n-1(2^n - 1)$ is a perfect number.
\end{proof}

\newpage
\subsection*{Problem 2}

\underline{Step 1:} Find $148^{-1}$ (mod 421)
\spacing

\noindent
Compute the steps of the Euclidean $GCD$ algorithm:

\begin{align*}
  421 &= 2 \times 148 + 125 \\
  148 &= 1 \times 125 + 23 \\
  125 &= 5 \times 23 + 10 \\
  23 &= 2 \times 10 + 3 \\
  10 &= 3 \times 3 + 1
\end{align*}

\spacing
\noindent
Roll back the steps to find $s, t \in \Z$ such that $1 = 421s + 148t$

\begin{align*}
  1 &= 1(10) - 3(3) \\
    &= 1(10) - 3(23 - 2(10)) \\
    &= 7(10) - 3(23) \\
    &= 7(125 - 5(23)) - 3(23) \\
    &= 7(125) - 38(23) \\
    &= 7(125) - 38(148 - 1(125)) \\
    &= 45(125) - 38(148) \\
    &= 45(421 - 2(148)) - 38(148) \\
    &= 45(421) - 128(148)
\end{align*}

\spacing
\noindent
So $1 = 45(421) - 128(148)$, we then get $1 = -128(148)$, 
hence $148^{-1}$ (mod 421) $\equiv -128 \equiv 293$ (mod 421).
\spacing

\noindent
\underline{Step 2}: Solve $148x \equiv 12$ (mod 421)

\begin{align*}
  148x &\equiv 12 \text{ (mod 421)} \\
     x &\equiv 12 \times 293 \text{ (mod 421)} \\
       &\equiv 3516 \text{ (mod 421)} \\
       &\equiv 148 \text{ (mod 421)}
\end{align*}

\newpage
\subsection*{Problem 3}

Recall: $a^{p - 1} \equiv 1 \text{ (mod p)}$

\subsubsection*{(a)}

\begin{align*}
  2409^{1335} \text{ (mod 19)} &= 2409^{(18 \cdot 74) + 3} \text{ (mod 19)} \\
    &= 2409^3 \text{ (mod 19)} && \text{[$2409 \equiv 15$ (mod 19)]} \\
    &= 15^3 \text{ (mod 19)} \\
    &= 3375 \text{ (mod 19)} \\
    &= 12 \text{ (mod 19)}
\end{align*}

\subsubsection*{(b)}

\begin{align*}
  7^{42806} \text{ (mod 349)} &= 7^{(123 \cdot 348) + 2} \text{ (mod 349)} \\
    &= 7^2 \text{ (mod 349)} \\
    &= 49 \text{ (mod 349)}
\end{align*}

\newpage
\subsection*{Problem 4}

\subsubsection*{(a)}

\begin{align*}
  \hat{M} &= M^p \text{ mod n} \\
    &= 4^5 \text{ mod 91} \\
    &= 1024 \text{ mod 91} \\
    &= 23
\end{align*}

\subsubsection*{(b)}

\begin{align*}
  x &= p^{-1} \text{ mod $(q_{1} - 1)(q_{2} - 1)$} \\
    &= 5^{-1} \text{ mod 72}
\end{align*}

\spacing
\noindent
Compute the steps of the Euclidean $GCD$ algorithm:

\begin{align*}
  72 &= 14 \times 5 + 2 \\
  5 &= 2 \times 2 + 1
\end{align*}

\spacing
\noindent
Roll back the steps to find $s, t \in \Z$ such that $1 = 5s + 72t$

\begin{align*}
  1 &= 5 - 2(2) \\
    &= 5 - 2(72 - 14(5)) \\
    &= 29(5) - 2(72)
\end{align*}

\noindent
So we see that $x = 29$, since $p \cdot 29 \equiv 1$ mod $72$.

\subsubsection*{(c)}

\begin{align*}
  M &= \hat{M}^x \text{ mod n} \\
    &= \hat{M}^{29} \text{ mod 91} \\
    &= 23^{29} \text{ mod 91} \\
    &= 4
\end{align*}

\end{document}